%%%%%%%%%%%%%%%%%%%%%%%%%%%%%%%%%%%%%%%%%%%%%%%%%%%%%%
%
%	Preamble
%
%%%%%%%%%%%%%%%%%%%%%%%%%%%%%%%%%%%%%%%%%%%%%%%%%%%%%%
\documentclass[12pt,english]{article}

%%%%%%%%%%%%%%%%%%%%%%%%%%%%%%%%%%%%%%%%%%
%	Include relevant packages
%%%%%%%%%%%%%%%%%%%%%%%%%%%%%%%%%%%%%%%%%%
\usepackage[margin=1in,paperwidth=8.5in,paperheight=11in,marginpar=0in]{geometry}
\usepackage{amsmath}
\usepackage{amssymb}
\usepackage{bm}
\usepackage{babel}
\usepackage{hyperref}
\hypersetup{
    unicode=false,          % non-Latin characters in Acrobat�s bookmarks
    bookmarks=false,        % show Acrobat�s Bookmarks?
    pdfstartview=FitH,
    pdftitle={Latent Factor Model for Momentum - Bayesian Version},    % title
    pdfauthor={Matthew Yeaton},          % author
    pdfsubject={Micro},       % subject of the document
    pdfnewwindow=true,                % links in new window
%    pdfkeywords={Urban Economics, Hedonics, Vacancy}, % list of keywords
%    pdftoolbar=true,       % show Acrobat�s toolbar?
%    pdfmenubar=true,       % show Acrobat�s menu?
    colorlinks=true,        % false: boxed links; true: colored links
    linkcolor=black,        % color of internal links
    citecolor=black,        % color of links to bibliography
    filecolor=black,        % color of file links
    urlcolor=black          % color of external links
}

%%%%%%%%%%%%%%%%%%%%%%%%%%%%%%%%%%%%%%%%%%
%	New commands
%%%%%%%%%%%%%%%%%%%%%%%%%%%%%%%%%%%%%%%%%%

\newcommand{\ib}{\begin{itemize}}
\newcommand{\ie}{\end{itemize}}
\newcommand{\eb}{\begin{enumerate}}
\newcommand{\ee}{\end{enumerate}}

\setlength{\parskip}{.5em}

%%%%%%%%%%%%%%%%%%%%%%%%%%%%%%%%%%%%%%%%%%%%%%%%%%%%%%
%
%	Document body
%
%%%%%%%%%%%%%%%%%%%%%%%%%%%%%%%%%%%%%%%%%%%%%%%%%%%%%%
\begin{document}

\title{Latent Factor Model for Momentum\\Bayesian Version}
% \title{Statement of Purpose - Draft \author{Matthew Yeaton\thanks{Matthew Yeaton, \href{mailto:myeaton1@gmail.com}{myeaton1@gmail.com}.}}}
\date{}
% \date{\today}
\maketitle

\section{Model}

State-space representation:

Observation equation: $$r_t = H'\xi_t + e_t = [0 \quad B \quad B]' \begin{pmatrix} \lambda_t \\ \lambda_{t-1} \\ f_t \end{pmatrix} + e_t$$

$$e_t \sim \mathcal{N}(0,\Sigma_e), \Sigma_e \text{ diagonal}$$

Transition equation: $$\xi_t = \alpha + F\xi_{t-1} + \omega_t \Rightarrow \begin{pmatrix} \lambda_t \\ \lambda_{t-1} \\ f_t \end{pmatrix} = \begin{bmatrix} (I - \Phi)\mu \\ 0 \\ 0 \end{bmatrix} + \begin{bmatrix} \Phi & 0 & 0 \\ I_K & 0 & 0 \\ 0 & 0 & 0 \end{bmatrix} \begin{pmatrix} \lambda_{t-1} \\ \lambda_{t-2} \\ f_{t-1} \end{pmatrix} + \begin{pmatrix} v_t \\ 0 \\ f_t \end{pmatrix}$$

$$\omega_t \sim \mathcal{N}(0,\boldsymbol{\Sigma}), \boldsymbol{\Sigma} = \begin{bmatrix} \Sigma_v & 0 & 0 \\ 0 & 0 & 0 \\ 0 & 0 & \Sigma_f \end{bmatrix}, \Sigma_v, \Sigma_f \text{ diagonal}$$

\section{Procedure}

First, initialize $\{\xi_t\}$ using principal components and data available for the entire sample.  Based on these estimates of $\{\xi_t\}$, get initial values of $F, H, \Sigma_e$ and $\boldsymbol{\Sigma}$ (specifically $\Phi, B, \Sigma_e, \Sigma_v, \Sigma_f$). Then each iteration of the Gibbs sampler is as follows:

\eb

    \item Conditional on $F, H, \Sigma_e$ and $\boldsymbol{\Sigma}$, draw $\{\xi_t\}$ using the Carter-Kohn (1994) procedure with the generalization that allows $\boldsymbol{\Sigma}$ to be singular.

    \item Conditional on $\{\xi_t\}$, draw $F, H, \Sigma_e$ and $\boldsymbol{\Sigma}$.

    \item Data augmentation: Conditional on $\{\xi_t\}, H$ and $\Sigma_e$, sample $\{r_t\}$ for those with missing values.

\ee

\end{document}